\documentclass[12pt,titlepage,twoside,openright]{book}

% Codificación y lenguaje
\usepackage[T1]{fontenc}
\usepackage[utf8]{inputenc}
\usepackage[spanish]{babel}
\usepackage{enumitem}
\usepackage{tocloft}
\newcommand{\listequationsname}{Índice de fórmulas}
\newlistof{myequations}{eq}{\listequationsname}
\addto\captionsspanish{\renewcommand{\listtablename}{Índice de tablas}}
\addto\captionsspanish{\renewcommand{\listfigurename}{Índice de figuras}}
\addto\captionsspanish{\renewcommand{\listequationsname}{Índice de fórmulas}}

% Tipografía
\usepackage{lmodern}

% Paquetes gráficos y matemáticos
\usepackage{graphicx}
\usepackage{color}
\usepackage{pstricks, pst-node}
\usepackage{amsfonts, amsmath, amssymb, amsthm}

% Encabezados
\usepackage{fancyhdr}
\pagestyle{fancy}
\fancyhead{} % Limpia encabezados
\fancyfoot{} % Limpia pies
\fancyhead[LE,RO]{\thepage}
\fancyhead[RE]{\leftmark}
\fancyhead[LO]{\rightmark}
\renewcommand{\headrulewidth}{0.4pt}
\renewcommand{\footrulewidth}{0pt}

% Hipervínculos y URLs
\usepackage{hyperref}

% Márgenes
\usepackage{geometry}
\geometry{
    left=3cm,
    right=2.5cm,
    top=2.5cm,
    bottom=2.5cm
}
\setlength{\headheight}{15pt}

% Bibliografía
\usepackage{natbib}

% Configuración hipervínculos
\hypersetup{
    colorlinks=true,
    linkcolor=black,
    citecolor=black,
    urlcolor=black,
    pdftitle={VRPTW: Proyecto Terminal},
    pdfauthor={Alejandro Martínez Guzmán}
}

% Páginas en blanco sin número ni encabezado
\makeatletter
\def\cleardoublepage{\clearpage\if@twoside \ifodd\c@page\else
\hbox{}
\thispagestyle{empty}
\newpage
\if@twocolumn\hbox{}\newpage\fi\fi\fi}
\makeatother

\hyphenation{Colores fa-mi-lia caracte-rizarlos}

% Entornos matemáticos
\newtheorem{teo}{Teorema}[chapter]
\newtheorem{defi}{Definición}[chapter]
\newtheorem{coro}{Corolario}[chapter]
\newtheorem{lema}{Lema}[chapter]
\newtheorem{prop}{Proposición}[chapter]
\newtheorem{ejemplo}{Ejemplo}[chapter]
\newtheorem{obs}{Observación}[chapter]
\newenvironment{demo}{
  \par\noindent\textbf{Demostración:}\quad
}{
  \hfill$\square$\par
}

% Comandos para portada
\newcommand{\titulo}[1]{\def\eltitulo{#1}}
\newcommand{\carrera}[1]{\def\lacarrera{#1}}
\newcommand{\nombre}[1]{\def\elnombre{#1}}
\newcommand{\director}[1]{\def\eldirector{#1}}
\newcommand{\fecha}[1]{\def\lafecha{#1}}

\titulo{Resolución del Problema de Rutas Vehiculares con Capacidades y Ventanas de Tiempo mediante un Algoritmo Híbrido entre Colonia de Hormigas y Recocido Simulado}
\nombre{\uppercase{Alejandro Martínez Guzmán}}
\carrera{Licenciatura en Ingeniería en Computación}
\director{\uppercase{Edwin Montes Orozco}}
\fecha{Junio 2025}


\begin{document}

%%%%%%%%%%%%%%%%%%%%%%%%%%%%%%%%%%%%%%%%%%%%%%%%%%%%%%%
% PORTADA
%%%%%%%%%%%%%%%%%%%%%%%%%%%%%%%%%%%%%%%%%%%%%%%%%%%%%%%
\begin{titlepage}
	\thispagestyle{empty}
	\hskip-1.5cm
	\begin{minipage}[c][5cm][s]{4cm}
		\begin{center}
			\hskip2pt \vrule width2pt height16cm\hskip1mm
			\vrule width1pt height16cm\\[10pt]
		\end{center}
	\end{minipage}\quad
	\begin{minipage}[c][9.5cm][s]{10cm}
		\begin{center}
			{\Large \scshape Universidad Autónoma Metropolitana}
			\vspace{.5cm}\hrule height2pt\vspace{.1cm}\hrule height1pt
			\vspace{.5cm}{\scshape UNIDAD CUAJIMALPA}\vspace{.5cm}
			\\
			\includegraphics[width=3cm]{Imagnes/uamlogo.png}\vspace{.5cm}
			\\
			{\Large \textsc{\eltitulo}}\vspace{1cm}
			\makebox[8cm][c]{\Large Proyecto Terminal}\vspace{1cm}
			\makebox[6cm]{QUE PRESENTA:}\\[3pt]\elnombre\vspace{1cm}
			{\textsc{\large \lacarrera}}\vspace{1cm}\\
			Departamento de Matemáticas Aplicadas e Ingeniería\\[13pt]
			División de Ciencias Naturales e Ingeniería\vspace{.5cm}
			\\
			Asesor:\\ \eldirector\vfill
			\begin{flushright}\lafecha\end{flushright}
		\end{center}
	\end{minipage}
\end{titlepage}
\cleardoublepage

%%%%%%%%%%%%%%%%%%%%%%%%%%%%%%%%%%%%%%%%%%%%%%%%%%%%%%%
% PÁGINAS PRELIMINARES SIN NÚMERO
%%%%%%%%%%%%%%%%%%%%%%%%%%%%%%%%%%%%%%%%%%%%%%%%%%%%%%%
\frontmatter

\chapter*{Declaración}
\thispagestyle{empty}
Yo, \elnombre, declaro que este trabajo titulado
\textit{«\eltitulo»} es de mi autoría. Asimismo, confirmo que:
\begin{itemize}
	\item Este trabajo fue realizado en su totalidad para la obtención de grado en esta Universidad.
	\item Ninguna parte de esta tesis ha sido previamente sometida a un examen de grado o titulación en esta u otra institución.
	\item Todas las citas han sido debidamente referenciadas y atribuidas a sus autores.
\end{itemize}
\vfill
\begin{flushright}
	Firma: \underline{\hspace{5cm}}\\[0.5cm]
	Fecha: \underline{\hspace{5cm}}
\end{flushright}
\cleardoublepage

\chapter*{Resumen}
\addcontentsline{toc}{chapter}{Resumen}
Aquí va el resumen en español. Este apartado debe sintetizar brevemente el objetivo del trabajo, la metodología empleada y los resultados más relevantes.
\cleardoublepage

\chapter*{Abstract}
\addcontentsline{toc}{chapter}{Abstract}
Here goes the abstract in English. Briefly describe the goal of your project, methodology, and key results.
\cleardoublepage

\chapter*{Dedicatoria}
\addcontentsline{toc}{chapter}{Dedicatoria}
\begin{flushright}
	\textit{A mis padres y profesores, por su apoyo incondicional.}
\end{flushright}
\cleardoublepage

\chapter*{Agradecimientos}
\addcontentsline{toc}{chapter}{Agradecimientos}
Aquí van los agradecimientos a las personas e instituciones que contribuyeron al desarrollo de este proyecto.
\cleardoublepage

%%%%%%%%%%%%%%%%%%%%%%%%%%%%%%%%%%%%%%%%%%%%%%%%%%%%%%%
% ÍNDICES
%%%%%%%%%%%%%%%%%%%%%%%%%%%%%%%%%%%%%%%%%%%%%%%%%%%%%%%

\cftsetindents{subsection}{2em}{3em}
\cftsetindents{subsubsection}{4em}{3em}
\setcounter{tocdepth}{3}
\setcounter{secnumdepth}{3}


\tableofcontents
\cleardoublepage
\listoffigures
\cleardoublepage
\listoftables
\cleardoublepage
\listofmyequations
\cleardoublepage


%%%%%%%%%%%%%%%%%%%%%%%%%%%%%%%%%%%%%%%%%%%%%%%%%%%%%%%
% CUERPO PRINCIPAL (ya numerado)
%%%%%%%%%%%%%%%%%%%%%%%%%%%%%%%%%%%%%%%%%%%%%%%%%%%%%%%
\mainmatter
\pagestyle{fancy}
\fancyhf{}
\fancyhead[RO,LE]{\bfseries \thepage}
\fancyhead[LO]{\nouppercase{\rightmark}}
\fancyhead[RE]{\nouppercase{\leftmark}}
\fancyfoot{}

\setlength{\parindent}{0pt}
\setlength{\parskip}{1.5ex}

\newcommand{\tab}{\hspace*{1cm}}

\chapter{Introducción}
\label{cap:introduccion}
% Contenido...

\chapter{Marco Teórico}
\label{cap:marco-teorico}
\section{Antencedentes}
\section{Conceptos Clave}
\subsection{El Problema en Optimización}

Los \textbf{problemas de optimización} constituyen una herramienta fundamental en ciencias computacionales, ingeniería, logística y muchas otras disciplinas. En el contexto de esta investigación, resultan especialmente relevantes porque permiten modelar situaciones en las que se busca obtener el \textbf{mejor resultado posible} bajo ciertas restricciones, como \textbf{minimizar costos}, \textbf{distancias} o \textbf{tiempos}.

De acuerdo con \citep{cobos2010}, los problemas de optimización se dividen de manera natural en dos categorías: problemas con \textbf{variables continuas} y problemas con \textbf{variables discretas}. A estos últimos se les conoce como problemas de \textbf{optimización combinatoria}.

La función \(f : S \to \mathbb{R}\), donde el conjunto \(S\) es un subconjunto de \(\mathbb{R}^n\), se denomina \textbf{función objetivo}, \textbf{función de costo} o \textbf{beneficio}, mientras que el conjunto \(S\) se identifica como el \textbf{conjunto factible} o el conjunto de \textbf{soluciones posibles} \citep{cobos2010}.

Como definición formal, se puede decir que el problema general de optimización consiste en encontrar un elemento \(x \in S\) que \textbf{optimice} la función objetivo. En el caso de un problema de \textbf{minimización}, se expresa de la siguiente manera:

\[
	\min_{x \in S} f(x).
\]

Por su parte, en un problema de \textbf{maximización}, se utiliza la notación:

\[
	\max_{x \in S} f(x).
\]

\citep{cobos2010}



\subsubsection{Optimización Continua}

En la \textbf{optimización continua} para problemas de minimización, el objetivo es encontrar un punto \( x_{opt} \) dentro del conjunto factible \( S \) tal que el valor de la función objetivo en ese punto sea menor o igual que el valor en cualquier otro punto del conjunto, es decir,

\[
	f(x_{opt}) \leq f(x), \quad \forall x \in S.
\]

Para problemas de maximización, se busca un punto \( x_{opt} \) en \( S \) donde la función objetivo tome un valor mayor o igual al de cualquier otro punto factible, esto es,

\[
	f(x_{opt}) \geq f(x), \quad \forall x \in S.
\]

A este punto se le denomina \textbf{solución óptima global}, y el valor correspondiente \( f(x_{opt}) \) se conoce como \textbf{costo óptimo}. Además, el conjunto de todas las soluciones óptimas se representa por \( S_{opt} \) \citep{cobos2010}.


\subsubsection{Optimización Discreta o Combinatoria}
\label{subsec:opt_discreta}

Una instancia de un problema de \textbf{optimización combinatoria} puede representarse mediante una pareja \((S, f)\), donde \(S\) es un conjunto \textbf{finito} que contiene todas las soluciones posibles, y \(f\) es una función real, denominada \textbf{función objetivo} o \textbf{función costo}, definida como

\[
	f: S \to \mathbb{R}.
\]

Para problemas de minimización, se busca una solución \(i_{opt} \in S\) tal que

\[
	f(i_{opt}) \leq f(i), \quad \forall i \in S,
\]

mientras que para problemas de maximización, se requiere que

\[
	f(i_{opt}) \geq f(i), \quad \forall i \in S.
\]

La solución \(i_{opt}\) se denomina \textbf{solución óptima global}, y el valor \(f(i_{opt})\) representa el \textbf{costo óptimo}. El conjunto de todas las soluciones óptimas se denota como \(S_{opt}\). Un problema de optimización combinatoria se define entonces como el conjunto de todas sus instancias \citep{cobos2010}.

Comprender esta distinción es esencial para abordar el problema central de este trabajo, el cual se enmarca dentro de la optimización combinatoria. En este tipo de problemas, el número de soluciones posibles es \textbf{finito}, y el desafío consiste en identificar cuál de ellas representa la \textbf{mejor opción} según un criterio específico.

Esta base teórica permite introducir los problemas clásicos de enrutamiento, como el \textbf{Problema del Agente Viajero (TSP)}, el \textbf{Problema de Enrutamiento de Vehículos (VRP)} y sus variantes, como el \textbf{Problema de Enrutamiento de Vehículos con Ventanas de Tiempo (VRPTW)}, los cuales serán analizados más adelante por su relación directa con la planificación eficiente de rutas en contextos reales.

\subsection{Complejidad Computacional y Algorítmica}

El estudio de la complejidad en ciencias computacionales busca comprender qué tan difícil es resolver un problema, tanto desde su estructura teórica como desde los recursos requeridos para su solución. Esta dificultad se aborda desde dos enfoques: la \textbf{complejidad computacional}, que clasifica los problemas según el crecimiento de recursos necesarios en función del tamaño de la entrada, y la \textbf{complejidad algorítmica}, que evalúa el desempeño de algoritmos específicos al aplicarse sobre dichas entradas \citep{garey1979computers, papadimitriou1994}.

De forma simplificada, la complejidad computacional se refiere al análisis general del problema sin importar el algoritmo, mientras que la algorítmica se enfoca en el tiempo y espacio consumido por un algoritmo concreto \citep{maldonado2013problema}.

Esta distinción es clave al abordar problemas de optimización combinatoria como el \textbf{Problema del Agente Viajero (TSP)}, el \textbf{Problema de Ruteo de Vehículos (VRP)} y su variante principal en este trabajo: el \textbf{Problema de Enrutamiento de Vehículos con Ventanas de Tiempo (VRPTW)}. Estos problemas presentan una explosión combinatoria de soluciones conforme crece el número de clientes.

\subsection{El Problema P vs NP}

Todos los problemas, tanto en la vida como en la ciencia, pueden clasificarse en dos grupos principales: \textbf{problemas decidibles} y \textbf{problemas indecidibles}. Se dice que un problema es \textbf{\emph{indecidible}} cuando no existe ningún algoritmo que permita resolverlo, incluso si se dispusiera de tiempo y recursos ilimitados. En consecuencia, no es posible determinar si dicho problema se detendrá ni en qué momento lo hará. Por contraste, un problema es \textbf{\emph{decidible}} cuando existe, o al menos podría existir, algún algoritmo capaz de resolverlo \citep{maldonado2013problema}.

Además, un problema se considera decidible cuando es compresible; es decir, cuando existe una fórmula o algoritmo que permite condensarlo o expresarlo de forma finita. Por el contrario, los problemas indecidibles se consideran incompresibles debido a que no se pueden reducir a una descripción finita ni predecir su comportamiento computacional \citep{maldonado2013problema}.

\subsubsection{Clases de Complejidad Básicas}

Dentro de los problemas decidibles, se distinguen dos categorías fundamentales en la teoría de la complejidad: los problemas \textbf{\emph{P}} y los problemas \textbf{\emph{NP}} \citep{maldonado2013problema}.

\paragraph{Clase P}

Un problema pertenece a la clase \emph{P} si puede resolverse mediante un algoritmo que se ejecuta en tiempo polinomial con respecto al tamaño de la entrada. Es decir, el número de pasos que requiere el algoritmo está acotado por un polinomio de grado \(k\). De manera formal:

\[
	(\exists\,\alpha > 0)\,(\exists\,n_0 > 0)\;:\; D(n) \leq \alpha \cdot p(n) \quad (\forall\,n \geq n_0),
\]

donde \(p(n)\) es un polinomio. En notación asintótica, esto se expresa como:

\[
	D = O(n^k).
\]

Aunque se dice que un algoritmo que se ejecuta en tiempo polinomial es “eficiente”, si el grado del polinomio es muy alto (por ejemplo, \(n^{10000}\)), su aplicabilidad práctica puede verse limitada. En general, se espera que los algoritmos eficientes tengan un grado bajo. Un ejemplo típico de problema en \emph{P} es la multiplicación de dos números naturales, cuyo algoritmo básico opera en tiempo lineal \citep{Flores2014}.

\paragraph{Clase NP}

Por otro lado, un problema pertenece a la clase \emph{NP} si, aunque no se sepa cómo resolverlo eficientemente, sí puede \emph{verificarse} en tiempo polinomial. Es decir, si se proporciona una posible solución (llamada certificado), puede comprobarse su validez en un tiempo razonable. Un ejemplo clásico es el ordenamiento de elementos: si bien existen algoritmos polinómicos para ordenarlos, el proceso de verificar si una secuencia está ordenada también toma tiempo lineal \citep{Flores2014}.

A pesar de que algunos algoritmos de ordenamiento, como el \emph{bubble sort}, operan en tiempo polinomial, en teoría podría existir un algoritmo de ordenamiento con tiempo exponencial. Sin embargo, si al menos uno de los algoritmos conocidos resuelve el problema en tiempo polinomial, el problema pertenece a la clase \emph{P} \citep{Flores2014}.

\paragraph{La gran incógnita: ¿P = NP?}

Una de las cuestiones abiertas más fundamentales en ciencias computacionales es determinar si \(\mathbf{P = NP}\) o \(\mathbf{P \neq NP}\). Esta pregunta implica saber si todo problema cuya solución puede verificarse rápidamente, también puede resolverse rápidamente. Es decir, si los problemas que se pueden verificar en tiempo polinomial también pueden resolverse en ese mismo tipo de tiempo. Este dilema sigue sin resolverse y tiene profundas implicaciones teóricas y prácticas \citep{maldonado2013problema}.

\subsubsection{NP-Completo y NP-Difícil}

Dentro de los problemas de clase \emph{NP}, existe una subcategoría especialmente importante: los problemas \textbf{NP-completos}. Estos son problemas que pertenecen a \emph{NP} y que, además, son al menos tan difíciles como cualquier otro problema en \emph{NP}. En otras palabras, si se lograra resolver uno de estos problemas de forma eficiente, también podrían resolverse todos los demás problemas en \emph{NP} \citep{maldonado2013problema}.

Por otra parte, los \textbf{problemas NP-difíciles (NP-hard)} son aquellos que pueden ser incluso más complejos que los \emph{NP-completos}, ya que no necesariamente pertenecen a \emph{NP} (es decir, podrían no tener soluciones verificables en tiempo polinomial), pero son al menos tan difíciles como los problemas más complicados de \emph{NP}. Este tipo de problemas puede presentarse como problemas de decisión, de búsqueda o de optimización \citep{maldonado2013problema}.

El crecimiento exponencial del tiempo requerido para resolver este tipo de problemas, como el \textbf{Problema del Agente Viajero (TSP)}, el \textbf{Problema de Ruteo de Vehículos (VRP)} y el \textbf{Problema de Ruteo de Vehículos con Ventanas de Tiempo (VRPTW)}, con respecto al tamaño de la entrada, hace que su solución exacta sea impráctica en muchos casos.

Por esta razón, en la práctica se recurre al uso de \textbf{algoritmos exactos} o \textbf{métodos de aproximación}, los cuales permiten encontrar soluciones suficientemente buenas en tiempos razonables, especialmente cuando se trata de instancias grandes o de elevada complejidad.

\subsection{Problemas de Optimización Combinatoria}

De manera complementaria a la definición formal presentada en la \autoref{subsec:opt_discreta} por \citep{cobos2010}, Papadimitriou y Steiglitz \citep{papadimitriou1998} indica la optimización combinatoria como el estudio de problemas de optimización en los que el conjunto de soluciones factibles es discreto o puede hacerse discreto mediante un proceso de enumeración, y en los que se busca una solución que optimice una función objetivo definida sobre ese conjunto.

Los problemas de optimización combinatoria constituyen una clase amplia de modelos que encuentran aplicación en numerosas áreas, especialmente en la planificación y enrutamiento de vehículos. Dentro de este campo, destacan el \textbf{Problema del Agente Viajero (TSP)}, que representa la forma más básica de la optimización de rutas, y distintas extensiones que incorporan restricciones adicionales propias de entornos reales, tales como las limitaciones de capacidad de los vehículos o las ventanas de tiempo en las que debe efectuarse la entrega de los productos. En este contexto, el \textbf{Problema de Ruteo de Vehículos con Ventanas de Tiempo (VRPTW)} se ha consolidado como uno de los desafíos más estudiados por su relevancia práctica y su elevada dificultad computacional. A continuación, se describen estos problemas en detalle:


\subsubsection{Problema del Agente Viajero (TSP)}
\label{subsec:problem_tsp}
El Problema del Agente Viajero, conocido por sus siglas en inglés como TSP (Travelling Salesman Problem), es uno de los problemas más reconocidos y complejos dentro de las ciencias computacionales. Ha sido estudiado desde diversas ramas de la ingeniería y por múltiples motivos. Su aplicación principal consiste en determinar rutas desde diferentes enfoques, ya sea en procesos que requieren una secuencia específica o en operaciones logísticas relacionadas con el transporte, con el objetivo de encontrar la ruta óptima considerando criterios de minimización de distancia o de costo \citep{lopez2014tabu}.

Según \citep{torres2018}, el Problema del Agente Viajero se define sobre una red \(G = [N,A,C]\), donde \(N\) es el conjunto de nodos, \(A\) es el conjunto de arcos y \(C = [c_{ij}]\) la matriz de costos (distancias) entre nodos \(i\) y \(j\). El objetivo es encontrar un ciclo hamiltoniano de costo mínimo, que recorra todos los nodos una sola vez y regrese al punto de partida.

\paragraph{Modelo Matemático del Agente Viajero (TSP)}

A continuación, se describe la formulación matemática más común:

\begin{equation}
	\min \sum_{i=1}^N \sum_{j=1}^N c_{ij}\,x_{ij}
	\label{eq:TSP_obj}
\end{equation}
\addcontentsline{eq}{myequations}{\protect\numberline{\theequation}Función objetivo TSP: minimizar el costo total del recorrido}

sujeto a:

\begin{equation}
	\sum_{j=1}^N x_{ij} = 1 \quad \forall i = 1,\dots,N
	\label{eq:TSP_out}
\end{equation}
\addcontentsline{eq}{myequations}{\protect\numberline{\theequation}Restricción de salida única de cada nodo para TSP}

\begin{equation}
	\sum_{i=1}^N x_{ij} = 1 \quad \forall j = 1,\dots,N
	\label{eq:TSP_in}
\end{equation}
\addcontentsline{eq}{myequations}{\protect\numberline{\theequation}Restricción de entrada única a cada nodo para TSP}

\begin{equation}
	\begin{cases}
		u_i - u_j + N x_{ij} \le N-1 & \forall i \neq j,\; i,j=2,\dots,N \\
		x_{ij} \in \{0,1\}           & \forall i,j
	\end{cases}
	\label{eq:TSP_subtour_bin}
\end{equation}
\addcontentsline{eq}{myequations}{\protect\numberline{\theequation}Restricción de eliminación de subciclos y variables binarias para TSP}

donde:
- \(x_{ij} = 1\) si la ruta va de \(i\) a \(j\), y 0 en caso contrario;
- \(u_i\) son variables auxiliares usadas para eliminar subciclos \citep{torres2018}.

La función objetivo~\eqref{eq:TSP_obj} proporciona la distancia total de los arcos que conforman un tour. Las restricciones en~\eqref{eq:TSP_out} garantizan que, al salir de una ciudad, sólo sea posible dirigirse a un único nodo. Las restricciones en~\eqref{eq:TSP_in} aseguran que a cada ciudad únicamente se llegue una sola vez. La restricción~\eqref{eq:TSP_subtour_bin} es fundamental en el planteamiento, pues mediante ella se evita que cualquier conjunto de \(x_{ij}\) que forme una subruta sea factible, de modo que sólo los conjuntos que constituyen un recorrido completo sean viables \citep{torres2018}.

\paragraph{Complejidad del TSP}

Como se explica en \cite{papadimitriou1998}, el Problema del Agente Viajero es NP-hard, y su versión de decisión es NP-completa. Esto implica que la búsqueda de un algoritmo eficiente para resolverlo es un desafío fundamental en la teoría de la computación, pues una solución polinomial para el TSP resolvería eficientemente toda la clase NP. Este carácter NP-hard justifica la utilización de algoritmos aproximados y metaheurísticos en su resolución, dado que las técnicas exactas sólo resultan prácticas para instancias de tamaño reducido.

\subsubsection{Problema del Problema Ruteo de Vehículos (VRP)}
El problema de ruteo de vehículos (VRP, por sus siglas en inglés) fue planteado por primera vez en 1959 por Dantzig y Ramser, quienes abordaron una aplicación práctica relacionada con la distribución de gasolina a estaciones de servicio, proponiendo además una formulación matemática del problema. Posteriormente, Clarke y Wright desarrollaron el primer algoritmo que demostró ser eficaz para resolverlo. A partir de ahí, se iniciaron numerosas investigaciones en el área del ruteo de vehículos. Este problema puede interpretarse como la convergencia de dos problemas clásicos de optimización combinatoria: el problema del agente viajero (TSP) mencionado en \autoref{subsec:problem_tsp}, en el que se asume una capacidad infinita de los vehículos, y el problema de empaquetamiento en compartimentos (BPP) \cite{daza2009}.

El Problema de Enrutamiento de Vehículos (VRP) se define como el problema de diseñar un conjunto óptimo de rutas para una flota de vehículos que parte desde un depósito central para servir a un conjunto dado de clientes, cada uno con una demanda específica. Cada ruta debe iniciarse y finalizar en el depósito, y cada cliente debe ser atendido exactamente una vez por un solo vehículo. El objetivo es minimizar el costo total del transporte, normalmente expresado como la distancia total recorrida por los vehículos \cite{toth2014}.

\paragraph{Modelo Matemático del Ruteo de Vehiculos (VRP)} A continuación, se describe la formulación matemática más común:

\subsection{Modelo matemático del Problema de Enrutamiento de Vehículos (VRP)}

Sea un grafo completo \( G = (V, E) \) donde \( V = \{0, 1, \ldots, n\} \) es el conjunto de nodos, con \(0\) representando el depósito y los nodos \(1, \ldots, n\) representando a los clientes. Se definen los parámetros y variables de decisión como sigue:

\begin{itemize}
	\item \( c_{ij} \geq 0 \): costo (distancia o tiempo) de ir del nodo \(i\) al nodo \(j\), para \(i,j \in V, i \neq j\).
	\item \( d_i \): demanda del cliente \(i\), para \(i=1,\ldots,n\).
	\item \( Q \): capacidad máxima de cada vehículo.
	\item \( K \): número máximo de vehículos disponibles.
	\item \( x_{ij} = \begin{cases} 1 & \text{si el arco } (i,j) \text{ es usado en la solución}\\ 0 & \text{en caso contrario} \end{cases} \), para \(i,j \in V, i \neq j\).
	\item \( u_i \): variable auxiliar para eliminación de subciclos y control de carga acumulada, para \(i=1,\ldots,n\).
\end{itemize}

\medskip

\textbf{Función objetivo:}

\begin{equation}
	\min \sum_{i \in V} \sum_{\substack{j \in V \\ j \neq i}} c_{ij} x_{ij}
\end{equation}

\medskip

\textbf{Sujeto a:}

\begin{align}
	 & \sum_{\substack{j \in V                                                                                                                                           \\ j \neq i}} x_{ij} = 1, && \forall i = 1, \ldots, n \quad \text{(cada cliente es visitado una vez)} \\
	 & \sum_{\substack{i \in V                                                                                                                                           \\ i \neq j}} x_{ij} = \sum_{\substack{k \in V \\ k \neq j}} x_{jk}, && \forall j = 1, \ldots, n \quad \text{(flujo conservado en clientes)} \\
	 & \sum_{j \in V \setminus \{0\}} x_{0j} \leq K, &  & \text{(número máximo de vehículos que salen del depósito)}                                                     \\
	 & u_i - u_j + Q x_{ij} \leq Q - d_j,            &  & \forall i,j \in V \setminus \{0\}, i \neq j \quad \text{(restricción de capacidad y eliminación de subciclos)} \\
	 & d_i \leq u_i \leq Q,                          &  & \forall i=1, \ldots, n                                                                                         \\
	 & u_0 = 0                                                                                                                                                           \\
	 & x_{ij} \in \{0,1\},                           &  & \forall i,j \in V, i \neq j
\end{align}


\paragraph{Complejidad del VRP}
Desde el enfoque de la complejidad computacional, el Problema de Enrutamiento de Vehículos (VRP) se considera altamente complejo, ya que pertenece a la clase de problemas NP-Completos. En el caso específico que se analiza, se han considerado diversas variantes del problema, ya que se busca cumplir múltiples objetivos en el proceso de enrutamiento y, adicionalmente, el modelo operativo del problema no es el más habitual, pues los repartos suelen organizarse de manera centralizada. Por esta razón, se ha llevado a cabo una revisión en la literatura enfocada en variantes como el MDVRP (VRP con múltiples depósitos), el VRPTW (VRP con ventanas de tiempo) y el CVRP (VRP con restricciones de capacidad), las cuales incorporan aspectos como la existencia de varios depósitos y restricciones temporales para la realización de las entregas correspondientes \cite{pino2011}.

\subsubsection{Problema de Ruteo de Vehículos con Ventanas de Tiempo (VRPTW)}




\subsection{Algoritmos Exactos}

\subsection{Heurísticas}

Heurística es un concepto cuyo origen se remonta a la Grecia clásica, derivado de la palabra griega \textit{heuriskein}, que significa encontrar o descubrir. Según la historia, se asocia con \textit{eureka}, la famosa exclamación atribuida a Arquímedes \citep{antonioSuarez2014}.

Para ejemplificar este concepto, se expone la siguiente definición:

\begin{quote}
	Según Zanakis et al.\ (citado en \citep{duarte2007metaheuristicas}), las heurísticas son \textit{``procedimientos simples, a menudo basados en el sentido común, que se supone que obtendrán una buena solución (no necesariamente óptima) a problemas difíciles de un modo sencillo y rápido''}.
\end{quote}

\subsubsection{Motivos para emplear métodos heurísticos}

Los problemas de decisión que pertenecen a la clase NP corresponden a aquellos para los que no se puede garantizar encontrar una mejor solución en un tiempo polinómico razonable.

En este contexto, los métodos heurísticos se convierten en procedimientos eficientes para hallar buenas soluciones, aunque no se pueda demostrar que estas sean óptimas. En estos métodos, la rapidez con que se obtiene el resultado (\textit{que siempre es menor que el tiempo requerido por otros métodos}) es tan relevante como la calidad de la solución encontrada.

Según \citep{antonioSuarez2014}, es posible emplear métodos heurísticos cuando un problema de optimización, sea determinístico o no, presenta alguna de las siguientes características:


\begin{enumerate}[label=\alph*.]
	\item El problema tiene una naturaleza tal que no se conoce ningún método exacto para resolverlo.
	\item Aunque exista un método exacto para su resolución, su uso resulta computacionalmente muy costoso o inviable.
	\item El método heurístico posee mayor flexibilidad que un método exacto, permitiendo, por ejemplo, incorporar condiciones de difícil modelización.
	\item El modelo matemático es demasiado extenso, excesivamente no lineal o muy complejo desde el punto de vista lógico.
	\item Realizar suposiciones o aproximaciones para simplificar el problema tiende a destruir estructuras del modelo que son esenciales en el contexto real, haciendo que la solución obtenida no sea viable.
	\item El método heurístico se emplea como parte de un procedimiento global que garantiza el óptimo de un problema. Existen dos posibilidades:
	      \begin{itemize}
		      \item El método heurístico proporciona una buena solución inicial de partida.
		      \item El método heurístico interviene en una etapa intermedia del procedimiento; un ejemplo de esto son las reglas que determinan la selección de la variable que entra en la base en el método Simplex.
	      \end{itemize}
\end{enumerate}

\subsubsection{Características de un buen algoritmo heurístico}

Según \citep{antonioSuarez2014}, un buen algoritmo heurístico debe poseer las siguientes propiedades:

\begin{enumerate}[label=\alph*.]
	\item \textbf{Ser eficiente}. Que el esfuerzo computacional sea realista y adecuado para obtener la solución.
	\item \textbf{Ser bueno}. La solución debe estar, en promedio, cerca del óptimo.
	\item \textbf{Ser robusto}. La probabilidad de obtener una mala solución (lejos del óptimo) debe ser baja.
\end{enumerate}

\subsection{Matehurísticas}

El término \textit{metaheurística} fue introducido por Fred Glover en 1986 \citep{antonioSuarez2014}. Etimológicamente, deriva de la composición de dos palabras de origen griego, que son “meta” y “heurística”. El segundo término ha sido descrito en la sección anterior, mientras que el prefijo meta puede traducirse como “más allá de” o “en un nivel superior” \citep{duarte2007metaheuristicas}.

Con este término, Glover pretendía definir un \textit{procedimiento maestro de alto nivel que guía y modifica otras heurísticas para explorar soluciones más allá de la simple optimalidad local} \citep{duarte2007metaheuristicas}.

Para ilustrar este concepto, cabe mencionar la definición propuesta por J.P. Kelly et al., quien es citado por \citep{duarte2007metaheuristicas}:

\begin{quote}
	\textit{Las metaheurísticas son métodos aproximados especialmente diseñados para abordar problemas complejos de optimización combinatoria donde los heurísticos tradicionales resultan ineficaces. Estas técnicas ofrecen un marco flexible para desarrollar algoritmos híbridos que integran conceptos provenientes de la inteligencia artificial, la evolución biológica y procesos estadísticos.}
\end{quote}

Las metaheurísticas constituyen un enfoque metodológico que permite el desarrollo de algoritmos híbridos innovadores mediante la integración de principios provenientes de múltiples disciplinas, incluyendo la genética, biología, inteligencia artificial, matemáticas, física y neurología \citep{antonioSuarez2014}.

\subsubsection{Criterios de Evaluación para Métodos Metaheurísticos}

La valoración del desempeño de las metaheurísticas se fundamenta en propiedades específicas que determinan su utilidad tanto práctica como teórica. Es importante reconocer que la optimización simultánea de todas estas características no es factible debido a la naturaleza contradictoria de algunas de ellas \citep{antonioSuarez2014}:

\begin{itemize}
	\item \textbf{Simplicidad:} La metodología debe fundamentarse en principios claros y accesibles que faciliten su comprensión.
	\item \textbf{Precisión:} Cada etapa y procedimiento debe estar definido mediante términos específicos y concretos.
	\item \textbf{Coherencia:} Los componentes metodológicos deben derivarse lógicamente de los principios fundamentales.
	\item \textbf{Eficiencia:} Debe optimizar el uso de recursos computacionales, tanto en tiempo de procesamiento como en memoria.
	\item \textbf{Generalidad:} La aplicabilidad debe extenderse a diversos tipos de problemas manteniendo un rendimiento satisfactorio.
	\item \textbf{Adaptabilidad:} Capacidad de ajustarse a diferentes escenarios de aplicación y modificaciones sustanciales del modelo.
	\item \textbf{Robustez:} El comportamiento debe mantenerse estable ante variaciones menores en el modelo o contexto.
	\item \textbf{Interactividad:} Debe facilitar la incorporación del conocimiento del usuario para optimizar el rendimiento.
	\item \textbf{Multiplicidad:} Capacidad de generar múltiples alternativas de solución de alta calidad para la selección del usuario.
	\item \textbf{Autonomía:} Funcionamiento independiente con parámetros mínimos o autoconfiguración.
\end{itemize}

\section{Referencias Relevantes}

\chapter{Materiales y Métodos}
\label{cap:materialesymetodos}
% Contenido...

\chapter{Resultados}
\label{cap:resultados}
% Contenido...

\chapter{Conclusiones}
\label{cap:conclusiones}
% Contenido...

%%%%%%%%%%%%%%%%%%%%%%%%%%%%%%%%%%%%%%%%%%%%%%%%%%%%%%%
% APÉNDICES Y REFERENCIAS
%%%%%%%%%%%%%%%%%%%%%%%%%%%%%%%%%%%%%%%%%%%%%%%%%%%%%%%
\backmatter

\bibliographystyle{plain}
\bibliography{referencias}

\appendix
\chapter{Apéndices}
\label{ap:apendices}
% Contenido...

\end{document}