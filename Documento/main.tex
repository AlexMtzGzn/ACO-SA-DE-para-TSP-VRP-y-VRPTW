\documentclass[12pt,titlepage,twoside,openright]{book}

% Codificación y lenguaje
\usepackage[T1]{fontenc}
\usepackage[utf8]{inputenc}
\usepackage[spanish]{babel}
\addto\captionsspanish{\renewcommand{\listtablename}{Índice de tablas}}%
\addto\captionsspanish{\renewcommand{\listfigurename}{Índice de figuras}}%

% Tipografía
\usepackage{lmodern}

% Paquetes gráficos y matemáticos
\usepackage{graphicx}
\usepackage{color}
\usepackage{pstricks, pst-node}
\usepackage{amsfonts, amsmath, amssymb, amsthm}

% Encabezados
\usepackage{fancyhdr}

% Hipervínculos y URLs
\usepackage{hyperref}
\usepackage{url}

% Márgenes
\usepackage{geometry}
\geometry{
    left=3cm,
    right=2.5cm,
    top=2.5cm,
    bottom=2.5cm
}
\setlength{\headheight}{15pt}

% Bibliografía
\usepackage{natbib}

% Configuración hipervínculos
\hypersetup{
    colorlinks=true,
    linkcolor=black,
    citecolor=blue,
    urlcolor=blue,
    pdftitle={VRPTW: Proyecto Terminal},
    pdfauthor={Alejandro Martínez Guzmán}
}

% Hacer que las páginas en blanco (por cleardoublepage) no tengan número ni encabezado
\makeatletter
\def\cleardoublepage{\clearpage\if@twoside \ifodd\c@page\else
\hbox{}
\thispagestyle{empty}
\newpage
\if@twocolumn\hbox{}\newpage\fi\fi\fi}
\makeatother

\hyphenation{Colores fa-mi-lia caracte-rizarlos}

% Entornos matemáticos
\newtheorem{teo}{Teorema}[chapter]
\newtheorem{defi}{Definición}[chapter]
\newtheorem{coro}{Corolario}[chapter]
\newtheorem{lema}{Lema}[chapter]
\newtheorem{prop}{Proposición}[chapter]
\newtheorem{ejemplo}{Ejemplo}[chapter]
\newtheorem{obs}{Observación}[chapter]
\newenvironment{demo}{\noindent\textbf{Demostración:}\ }{\hfill$\square$}

% Comandos para portada
\newcommand{\titulo}[1]{\def\eltitulo{#1}}
\newcommand{\carrera}[1]{\def\lacarrera{#1}}
\newcommand{\nombre}[1]{\def\elnombre{#1}}
\newcommand{\director}[1]{\def\eldirector{#1}}
\newcommand{\fecha}[1]{\def\lafecha{#1}}

\titulo{Resolución del Problema de Rutas Vehiculares con Ventanas de Tiempo mediante un Algoritmo Híbrido entre Colonia de Hormigas y Recocido Simulado}
\nombre{\uppercase{Alejandro Martínez Guzmán}}
\carrera{Licenciatura en Ingeniería en Computación}
\director{\uppercase{Edwin Montes Orozco}}
\fecha{Junio 2025}

\begin{document}

%%%%%%%%%%%%%%%%%%%%%%%%%%%%%%%%%%%%%%%%%%%%%%%%%%%%%%%
% PORTADA
%%%%%%%%%%%%%%%%%%%%%%%%%%%%%%%%%%%%%%%%%%%%%%%%%%%%%%%
\begin{titlepage}
	\thispagestyle{empty}
	\hskip-1.5cm
	\begin{minipage}[c][5cm][s]{4cm}
		\begin{center}
			\hskip2pt \vrule width2pt height16cm\hskip1mm
			\vrule width1pt height16cm\\[10pt]
		\end{center}
	\end{minipage}\quad
	\begin{minipage}[c][9.5cm][s]{10cm}
		\begin{center}
			{\Large \scshape Universidad Autónoma Metropolitana}
			\vspace{.5cm}\hrule height2pt\vspace{.1cm}\hrule height1pt
			\vspace{.5cm}{\scshape UNIDAD CUAJIMALPA}\vspace{.5cm}
			\\
			\includegraphics[width=3cm]{uamlogo.png}\vspace{.5cm}
			\\
			{\Large \textsc{\eltitulo}}\vspace{1cm}
			\makebox[8cm][c]{\Large Proyecto Terminal}\vspace{1cm}
			\makebox[6cm]{QUE PRESENTA:}\\[3pt]\elnombre\vspace{1cm}
			{\textsc{\large \lacarrera}}\vspace{1cm}\\
			Departamento de Matemáticas Aplicadas e Ingeniería\\[13pt]
			División de Ciencias Naturales e Ingeniería\vspace{1cm}
			\\
			Asesor:\\ \eldirector\vfill
			\begin{flushright}\lafecha\end{flushright}
		\end{center}
	\end{minipage}
\end{titlepage}
\cleardoublepage

%%%%%%%%%%%%%%%%%%%%%%%%%%%%%%%%%%%%%%%%%%%%%%%%%%%%%%%
% PÁGINAS PRELIMINARES SIN NÚMERO
%%%%%%%%%%%%%%%%%%%%%%%%%%%%%%%%%%%%%%%%%%%%%%%%%%%%%%%
\frontmatter

\chapter*{Declaración}
\thispagestyle{empty}
Yo, \elnombre, declaro que este trabajo titulado
\textit{«\eltitulo»} es de mi autoría. Asimismo, confirmo que:
\begin{itemize}
	\item Este trabajo fue realizado en su totalidad para la obtención de grado en esta Universidad.
	\item Ninguna parte de esta tesis ha sido previamente sometida a un examen de grado o titulación en esta u otra institución.
	\item Todas las citas han sido debidamente referenciadas y atribuidas a sus autores.
\end{itemize}
\vfill
\begin{flushright}
	Firma: \underline{\hspace{5cm}}\\[0.5cm]
	Fecha: \underline{\hspace{5cm}}
\end{flushright}
\cleardoublepage

\chapter*{Resumen}
\addcontentsline{toc}{chapter}{Resumen}
Aquí va el resumen en español. Este apartado debe sintetizar brevemente el objetivo del trabajo, la metodología empleada y los resultados más relevantes.
\cleardoublepage

\chapter*{Abstract}
\addcontentsline{toc}{chapter}{Abstract}
Here goes the abstract in English. Briefly describe the goal of your project, methodology, and key results.
\cleardoublepage

\chapter*{Dedicatoria}
\addcontentsline{toc}{chapter}{Dedicatoria}
\begin{flushright}
	\textit{A mis padres y profesores, por su apoyo incondicional.}
\end{flushright}
\cleardoublepage

\chapter*{Agradecimientos}
\addcontentsline{toc}{chapter}{Agradecimientos}
Aquí van los agradecimientos a las personas e instituciones que contribuyeron al desarrollo de este proyecto.
\cleardoublepage

%%%%%%%%%%%%%%%%%%%%%%%%%%%%%%%%%%%%%%%%%%%%%%%%%%%%%%%
% ÍNDICES
%%%%%%%%%%%%%%%%%%%%%%%%%%%%%%%%%%%%%%%%%%%%%%%%%%%%%%%
\tableofcontents
\cleardoublepage
\listoffigures
\cleardoublepage
\listoftables
\cleardoublepage

%%%%%%%%%%%%%%%%%%%%%%%%%%%%%%%%%%%%%%%%%%%%%%%%%%%%%%%
% CUERPO PRINCIPAL (ya numerado)
%%%%%%%%%%%%%%%%%%%%%%%%%%%%%%%%%%%%%%%%%%%%%%%%%%%%%%%
\mainmatter
\pagestyle{fancy}
\fancyhf{}
\fancyhead[RO,LE]{\bfseries \thepage}
\fancyhead[LO]{\nouppercase{\rightmark}}
\fancyhead[RE]{\nouppercase{\leftmark}}
\fancyfoot{}

\setlength{\parindent}{0pt}
\setlength{\parskip}{1.5ex}

\chapter{Introducción}
\label{cap:introduccion}
% Contenido...

\chapter{Marco Teórico / Estado del Arte}
\label{cap:marco-teorico}
\section{Antencedentes}
\section{Conceptos Clave}
\subsection{El Problema en Optimización}

La optimización es una rama de las matemáticas aplicadas que se ocupa de \textbf{determinar el valor máximo o mínimo de una función} que depende de una o más variables \citep{canek2008}.

Formalmente, un problema de optimización se define como la búsqueda de un vector \( x \) en un conjunto factible \( S \subseteq \mathbb{R}^n \) que minimice o maximice una función objetivo \( f: S \to \mathbb{R} \), también denominada función costo o beneficio \citep{cobos2010}.

Esta disciplina se clasifica principalmente en dos tipos: la \textit{optimización continua}, que trabaja con variables que pueden tomar valores dentro de un rango continuo, y la \textit{optimización discreta}, que se enfoca en problemas donde las variables sólo pueden asumir valores discretos o enteros \citep{cobos2010}.

\subsubsection{Optimización Continua}

La optimización continua se refiere a la búsqueda de soluciones óptimas en un conjunto factible \( S \subseteq \mathbb{R}^n \), donde las variables pueden tomar valores dentro de un intervalo continuo. En este contexto, el objetivo es encontrar un vector \( x_{opt} \in S \) que minimice o maximice una función objetivo \( f: S \to \mathbb{R} \).

Formalmente, para problemas de minimización, la solución óptima global \( x_{opt} \) satisface:

\[
f(x_{opt}) \leq f(x), \quad \forall x \in S,
\]

mientras que para maximización, se cumple que:

\[
f(x_{opt}) \geq f(x), \quad \forall x \in S.
\]

El valor \( f(x_{opt}) \) representa el costo o beneficio óptimo, y el conjunto de todas las soluciones óptimas se denota como \( S_{opt} \) \citep{cobos2010}.

\subsubsection{Optimización Discreta o Combinatoria}

Un problema de optimización combinatoria se formaliza mediante una pareja \((S, f)\), donde \(S\) es un conjunto finito de soluciones posibles y \(f: S \to \mathbb{R}\) es la función objetivo o función costo que asigna un valor real a cada solución \citep{cobos2010}.

El objetivo es encontrar una solución óptima global \(i_{opt} \in S\) que cumpla:

\[
f(i_{opt}) \leq f(i), \quad \forall i \in S,
\]

para problemas de minimización, o bien:

\[
f(i_{opt}) \geq f(i), \quad \forall i \in S,
\]

en el caso de maximización.


\section{Referencias Relevantes}

\chapter{Materiales y Métodos}
\label{cap:materialesymetodos}
% Contenido...

\chapter{Resultados}
\label{cap:resultados}
% Contenido...

\chapter{Conclusiones}
\label{cap:conclusiones}
% Contenido...

%%%%%%%%%%%%%%%%%%%%%%%%%%%%%%%%%%%%%%%%%%%%%%%%%%%%%%%
% APÉNDICES Y REFERENCIAS
%%%%%%%%%%%%%%%%%%%%%%%%%%%%%%%%%%%%%%%%%%%%%%%%%%%%%%%
\backmatter

\bibliographystyle{plain}
\bibliography{referencias}

\appendix
\chapter{Apéndices}
\label{ap:apendices}
% Contenido...

\end{document}