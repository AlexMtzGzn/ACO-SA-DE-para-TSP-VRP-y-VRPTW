\documentclass[12pt,titlepage,twoside,openright]{book}
\usepackage{color}
\usepackage{pstricks, pst-node}
\usepackage{graphics,graphicx,graphpap}
\usepackage{amsfonts}
\usepackage{amsmath}
\usepackage{amssymb}
\usepackage{amsthm}
\usepackage[ansinew]{inputenc}
\usepackage[spanish]{babel}
\usepackage{fancyhdr}
\pagestyle{empty}



%%%%%%%%%%%%%%%%%%%%%%%%%%%%%%%%%%%%%%%%%%%%%%%%%%%%%%%%
%%% Plantilla elaborada por Alma Rocío Sagaceta Mejía (2010) y modificada por Julián Alberto Fresán Figueroa (2011-2020) %%%
%%%%%%%%%%%%%%%%%%%%%%%%%%%%%%%%%%%%%%%%%%%%%%%%%%%%%%%


%%%%%%%%%%%%%%%%%%%%%%%%%%%%%%%%%%%%%%%%%%%%%%%%%%%%%%%
%Aqui se pueden agregar palabras que Latex separa de forma incorrecta al haber saltos de linea. Al escribir fa-mi-lia se le esta indicando que puede separar la palabra solo en los guiones
%%%%%%%%%%%%%%%%%%%%%%%%%%%%%%%%%%%%%%%%%%%%%%%%%%%%%%%

\hyphenation{Colores fa-mi-lia caracte-rizarlos}



%%%%%%%%%%%%%%%%%%%%%%%%%%%%%%%%%%%%%%%%%%%%%%%%%%%%
%%%              UAM-Logo                   	 %%%
%%%         Por Ismael Velázquez Ramírez         %%%
%%%%%%%%%%%%%%%%%%%%%%%%%%%%%%%%%%%%%%%%%%%%%%%%%%%%

\newcommand{\uamlogo}[3][2pt]{
    \psset{unit=#2,linewidth=#1 }
    \psline*[linearc=.25,linecolor=#3](2.8,2)(2,2)(1.8,0)(2.8,2)(3.8,0)(3.6,2)(2,2)(1.8,0)
    \psline*[linecolor=#3](0,0)(.8,0)(1.8,2)(1,2)(0,0)
    \psline*[linecolor=#3](4.8,0)(3.8,2)(4.6,2)(5.6,0)(4.8,0)
    \psline*[linearc=.25,linecolor=#3](3.8,0)(2.8,2)(3.6,2)(4.6,0)(3.8,0)
    \psline*[linearc=.25,linecolor=#3](4.6,0)(3.8,0)(2.8,2)(3.6,2)(4.6,0)
    \rput{180}(5.6,2){%
    \psline*[linearc=.25,linecolor=white](2.8,2)(2,2)(1.8,0)(2.8,2)(3.8,0)(3.6,2)(2,2)(1.8,0)
    \psline*[linearc=.25,linecolor=white](1,0)(1.8,0)(2.8,2)(2,2)(1,0)
    \psline*[linearc=.25,linecolor=white](1.8,0)(1,0)(2,2)(2.8,2)(1.8,0)
    \psline*[linearc=.25,linecolor=white](3.8,0)(2.8,2)(3.6,2)(4.6,0)(3.8,0)
    \psline*[linearc=.25,linecolor=white](4.6,0)(3.8,0)(2.8,2)(3.6,2)(4.6,0)
    \psline[linearc=.25,linecolor=#3](1,0)(2,2)(3.6,2)(4.6,0)
    \psline[linecolor=#3](1,0)(1.8,0)
    \psline[linearc=.25,linecolor=#3](4.6,0)(3.8,0)
    \psline[linearc=.25,linecolor=#3](1.8,0)(2.8,2)(3.8,0)}
    \psline*[linearc=.25,linecolor=#3](1,0)(1.8,0)(2.8,2)(2,2)(1,0)
    \psline*[linearc=.25,linecolor=#3](1.8,0)(1,0)(2,2)(2.8,2)(1.8,0)}
    
    
%%%%%%%%%%%%%%%%%%%%%%%%%%%%%%%%%%%%%%%%%%%%%%%%%%%%%%%
% Aqui se definen los entornos en español.
%%%%%%%%%%%%%%%%%%%%%%%%%%%%%%%%%%%%%%%%%%%%%%%%%%%%%%%

\newtheorem{teo}{Teorema}
\newtheorem{defi}{Definici\'on}
\newtheorem{coro}{Corolario}


%Ejemplo de uso:
%\begin{teo}[Teorema de Completitud] 
%Sea...
%\end{teo}



\newcommand{\titulo}[1]{\def\eltitulo{#1}}
\newcommand{\carrera}[1]{\def\lacarrera{#1}}
\newcommand{\nombre}[1]{\def\elnombre{#1}}    %* Del alumno
\newcommand{\director}[1]{\def\eldirector{#1}}  %* De tesis
\newcommand{\fecha}[1]{\def\lafecha{#1}}


%%%%%%%%%%%%%%%%%%%%%%%%%%%%%%%%%%%%%%%%%%%%%%%%%%%%%%%
% Se deben llenar los siguientes datos
%Estos datos apareceran en la portada y los encabezados
%%%%%%%%%%%%%%%%%%%%%%%%%%%%%%%%%%%%%%%%%%%%%%%%%%%%%%%

\titulo{Nombre del Proyecto}
\nombre{\uppercase{Nombre del alumno}}
\carrera{Licenciatura en ...}
\director{\uppercase{Nombre del asesor(es)}}
\fecha{Mes y a\~no (de finalizaci\'on)}


\begin{document}

%%%%%%%%%%%%%%%%%%%%%%%%%%%%%%%%%%%%%%%%%%%%%%%%%%%%%%%
% Portada
%%%%%%%%%%%%%%%%%%%%%%%%%%%%%%%%%%%%%%%%%%%%%%%%%%%%%%%


\thispagestyle{empty}

\hskip-1.5cm
\begin{minipage}[c][5cm][s]{4cm}
	\begin{center}
		%\rput(-2,.5){\uamlogo{.75}{black}}


		\hskip2pt \vrule width2pt height16cm\hskip1mm
		\vrule width1pt height16cm\\[10pt]

	\end{center}
\end{minipage}\quad
%% Barra derecha - Tí­tulos
\begin{minipage}[c][9.5cm][s]{10cm}
	\begin{center}
		% Barra superior
		{\Large \scshape Universidad Aut\'onoma Metropolitana}
		\vspace{.3cm}
		\hrule height2pt
		\vspace{.1cm}
		\hrule height1pt
		\vspace{.3cm}
		% \scshape
		{ UNIDAD CUAJIMALPA}


		%%%%%%%%%%%%%%%%%%%%%%%%%%%%%%%%%%%%%%%%%%%%%%%%%%%%%%%
		% Poner aquí el título del trabajo
		%%%%%%%%%%%%%%%%%%%%%%%%%%%%%%%%%%%%%%%%%%%%%%%%%%%%%%%
		\vspace{2cm}

		{\Large \textsc{GREEN-VRP}}

		\vspace{2cm}
		\makebox[8cm][c]{\Large Proyecto Terminal}\\[8pt]


		\makebox[6cm]{QUE PRESENTA:}\\[3pt]
		\elnombre\\
		\vspace{.5cm}
		{\textsc {\large \lacarrera}}\\[3pt]
		\makebox[10cm]{Departamento de Matem\'aticas Aplicadas e Ingenier\'{i}a}\\[13pt]
		\makebox[10cm]{Divisi\'on de Ciencias Naturales e Ingenier\'ia}\\[13pt]


		\vspace{2cm}

		{ Asesor y Responsable de la tesis:\\ \eldirector\\}

		\vspace{2cm}
		\begin{flushright}
			\lafecha
		\end{flushright}

	\end{center}
\end{minipage}


\frontmatter{}



\pagestyle{plain}


\tableofcontents
\newpage
\listoffigures




\mainmatter{}
\pagestyle{fancy}
\renewcommand{\chaptermark}[1]{%
	\markboth{\chaptername
		\ \thechapter.\ #1}{}}
\fancyhead[RO,LE]{\bfseries \thepage}
\fancyfoot{}
\setlength{\parindent}{0pt}
\setlength{\parskip}{1.5ex}


%%%%%%%%%%%%%%%%%%%%%%%%%%%%%%%%%%%%%%%%%%%%%%%%%%%%%%%
% Aquí se empieza a colocar la información del trabajo
% El documento deberá contener al menos los  capítulos que aparecen a continuación.
% Cada capitulo puede tener secciones y subsecciones
% Las imágenes deben ser incluidas en formato .eps
%%%%%%%%%%%%%%%%%%%%%%%%%%%%%%%%%%%%%%%%%%%%%%%%%%%%%%%

\chapter{Resumen}

%Uno o dos párrafos de que se hizo en el proyecto terminal


\chapter{Introducci\'on}
%una o dos p\'aginas, escrita para ser comprensible por cualquier persona
\chapter{Conocimientos preliminares}
%En este capítulo deberán incluir todos los conocimientos necesarios para poder entender los temas que se trataron en el proyecto. 
\section{Problema de Optimizaci\'on}

La optimizaci\'on es una rama de las matem\'aticas aplicadas que se enfoca en el desarrollo de principios y m\'etodos para resolver problemas cuantitativos en diversas disciplinas como la f\'isica, biolog\'ia, ingenier\'ia o econom\'ia, as\'i como en cualquier campo donde sea necesaria la toma de decisiones dentro de un conjunto de opciones, con el objetivo de encontrar la mejor o una de las mejores soluciones de manera eficiente.

En t\'erminos formales, consideramos una funci\'on \( f : S \rightarrow \mathbb{R} \), donde \( S \subseteq \mathbb{R}^n \). Denominaremos a \( f \) como funci\'on objetivo, y a \( S \) lo llamaremos conjunto factible o conjunto de soluciones posibles.

\section{Tipos de Optimizaci\'on}

Dentro de la optimizaci\'on, existen dos tipos principales: la optimizaci\'on discreta y la optimizaci\'on continua.

La optimizaci\'on discreta se aplica cuando el dominio \( S \) de la funci\'on objetivo es un conjunto discreto. En este contexto, el conjunto de soluciones posibles es finito o numerablemente infinito. La optimizaci\'on discreta a menudo involucra la b\'usqueda de soluciones en combinaciones espec\'ificas y puede involucrar problemas como la programaci\'on lineal entera o los problemas de asignaci\'on. Las soluciones \'optimas pueden ser dif\'iciles de encontrar debido a la naturaleza combinatoria del problema.

Por otro lado, la optimizaci\'on continua se refiere a situaciones en las que el conjunto \( S \) de la funci\'on objetivo es un conjunto continuo. En este caso, el dominio es un intervalo o un subconjunto de \( \mathbb{R}^n \) que no es discreto. Aqu\'i se buscan soluciones que maximizan o minimizan la funci\'on objetivo sobre un espacio continuo, y los m\'etodos comunes incluyen la programaci\'on lineal, la programaci\'on no lineal y el c\'alculo de variaciones. La optimizaci\'on continua suele implicar el uso de t\'ecnicas de c\'alculo y an\'alisis matem\'atico.

%\subsection{Optimizaci\'on Discreta}

%\subsection{Optimizaci\'on continua}

\section{Heur\'istica y Metaheur\'istica}

La palabra ``heur\'istica'' proviene del t\'ermino griego ``euriskein'', que significa ``encontrar''.
En el contexto de la optimizaci\'on, se refiere a t\'ecnicas dise\~nadas para mejorar o resolver problemas que, de otra manera, no tendr\'ian una soluci\'on eficiente. Los algoritmos heur\'isticos ayudan a encontrar soluciones aproximadas para problemas cuyas soluciones exactas no son factibles en tiempos polinomiales.
Muchos algoritmos de inteligencia artificial son heur\'isticos o se basan en sus principios para resolver problemas.

El prefijo ``meta'' otorga un sentido de mayor nivel a la palabra heur\'istica, lo que permite abordar problemas de mayor complejidad mediante soluciones factibles.

Tanto las heur\'isticas como las metaheur\'isticas se pueden utilizar de manera intercambiable para resolver problemas de optimizaci\'on combinatoria.

\section{Complejidad P, NP, NP-completo y NP-dif\icil.}
\chapter{Desarrollo del proyecto}
%En Este capítulo se incluirán los resultados obtenidos durante el proyecto

\chapter{Conclusiones y trabajo futuro}
%En este capítulo se incluirán las conclusiones y posibles líneas de trabajo que podrían surgir a partir de este proyecto


%%%%%%%%%%%%%%%%%%%%%%%%%%%%%%%%%%%%%%%%%%%%%%%%%%%%%%%
% La bibliografía debe aparecer de manera homologada
%%%%%%%%%%%%%%%%%%%%%%%%%%%%%%%%%%%%%%%%%%%%%%%%%%%%%%%

\begin{thebibliography}{0}
	%\bibitem{}

\end{thebibliography}

\end{document}


